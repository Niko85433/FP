%----------------------------------------------------------------------------------------
%	PACKAGES AND OTHER DOCUMENT CONFIGURATIONS
%----------------------------------------------------------------------------------------

\documentclass[DIV=calc, paper=a4, fontsize=11pt, twocolumn, fleqn]{scrartcl}	 % A4 paper and 11pt font size

%\usepackage[ngerman]{babel} % English language/hyphenation
\usepackage[utf8]{inputenc}
%\usepackage[protrusion=true,expansion=true]{microtype} % Better typography
\usepackage[svgnames]{xcolor} % Enabling colors by their 'svgnames'
\usepackage[hang, small, labelfont=bf, up, textfont=it]{caption} % Custom captions under/above floats in tables or figures
\usepackage{booktabs} % Horizontal rules in tables
\usepackage{fix-cm}	 % Custom font sizes - used for the initial letter in the document
\renewcommand\thesection{\Roman{section}} % Roman numerals for the sections

\usepackage{sectsty} % Enables custom section titles
\allsectionsfont{\centering\usefont{OT1}{phv}{b}{it}} % Change the font of all section commands

%\usepackage{fancyhdr} % Needed to define custom headers/footers
%\pagestyle{fancy} % Enables the custom headers/footers
%\usepackage{lastpage} % Used to determine the number of pages in the document (for "Page X of Total")

\usepackage{kpfonts} %Font

%----------------------------------------------------------------------------------------
%	GRAPHICS
%----------------------------------------------------------------------------------------

\usepackage[space]{grffile}

\graphicspath{{C:/Users/Nikolas/SkyDrive/Uni/FP/ElektrischeSondenimPlasma/graphicx/}}

%----------------------------------------------------------------------------------------
%	MATH TOOLS
%----------------------------------------------------------------------------------------

\usepackage{amsmath,amsfonts,amsthm}
\usepackage{amssymb}
\usepackage{pifont}
\usepackage{mathtools}
%\usepackage{xfrac}
\usepackage{gensymb}
\usepackage{units}
\usepackage{graphicx}
\usepackage{enumerate}

%----------------------------------------------------------------------------------------
%	PLOTTING TOOLS
%----------------------------------------------------------------------------------------

\usepackage{gnuplottex}

%----------------------------------------------------------------------------------------
%	DRAWING TOOLS
%----------------------------------------------------------------------------------------

%\usepackage{tikz}
%\usepackage{pstricks,pst-text}
%\usepackage{etex,pst-plot}
%\usepackage{pst-coil}
%\usepackage[crop=off]{auto-pst-pdf}
%\usepackage{epstopdf}

%----------------------------------------------------------------------------------------
%	NEW DEFINITIONS
%----------------------------------------------------------------------------------------

\newcommand{\mathtext}[1]{\text{\scriptsize{#1}}}
\newcommand{\subsc}[2]{#1_{\text{\scriptsize #2}}}
\renewcommand{\d}{\text{d}}
\newcommand{\md}{\scriptsize{.}}
\newcommand{\mc}{\scriptsize{,}}
\renewcommand{\figurename}{Fig.}
\renewcommand{\tablename}{Tab.}
\renewcommand{\labelitemi}{\ding{118}} % Item Symbol

%\addto\captionsenglish{
%  \renewcommand{\figurename}{Fig.}
%  \renewcommand{\tablename}{Table of Contents}
%}

%----------------------------------------------------------------------------------------
%	HEAD/FOOT
%----------------------------------------------------------------------------------------

\usepackage{scrlayer-scrpage}
\clearscrheadfoot
\ifoot{\footnotesize \rightmark}
\ofoot{\footnotesize \pagemark}

%\usepackage[autooneside=false,markcase=noupper]{scrlayer-scrpage}
%\clearscrheadfoot
%\ifoot{\ifstr{\rightbotmark}{\leftmark}{}{\rightbotmark}}
%\ofoot{\footnotesize \pagemark}

\automark[subsection]{section}

\pagestyle{scrheadings} % Show chapter titles as headings


%\makeatletter
%% Damit die letzte (\botmark) statt der ersten (\firstmark) Marke auf
%% einer Seite für die "rechte Marke" genommen wird:
%\providecommand*{\rightbotmark}{\expandafter\@rightmark\botmark\@empty\@empty}
%\makeatother

\setfootsepline{0.5pt}
\setheadsepline{0.5pt}

%----------------------------------------------------------------------------------------
%	ABSTRACT
%----------------------------------------------------------------------------------------

\usepackage{lettrine} % Package to accentuate the first letter of the text
\newcommand{\initial}[1]{ % Defines the command and style for the first letter
\lettrine[lines=3,lhang=0.3,nindent=0em]{
\color{RoyalBlue}
{\textsf{#1}}}{}}